% 从现在起,你将扮演一位科研工作者,有着丰富的论文写作经验。现在有学生向你请教论文写作的问题。可能会给你一段话,你将这段话用英文润色;也可能是直接请教你一些表达或科技论文的框架。你的最终目的是,指导该学生写出一篇优秀的、可发表的科技论文。

\documentclass{article}
\usepackage{graphicx}
\usepackage{subcaption}
\usepackage{enumerate}
\usepackage{amsmath}
\usepackage{multirow}
\usepackage{bm}
\usepackage[ruled,linesnumbered]{algorithm2e}

\def\T{\mathrm{T}}


\title{Greedy-based RBS maximum allowable current calculation method}
\author{3057761608 }
\date{March 2023}

\begin{document}

\maketitle

\section{Introduction}

% Battery energy storage systems(BESS) 被用在新能源汽车、风力发电站等场景中,为设备提供高品位电能的保存和释放。
Battery Energy Storage Systems (BESSs) are widely used to store and supply high-quality electrical energy in various applications, including electric vehicles and wind power turbines. 
Typically, a BESS consists of a large number of battery cells that are interconnected by series-parallel circuitry to provide the required charge storage capacity and output voltage. 
However, as the number of cells increases, the reliability of the system becomes a major concern. 
The capacity and safety of the BESS are mainly determined by the least healthy battery cells, a phenomenon known as the cask effect. 
Furthermore, the degradation of cells in poor condition is accelerated by multiple charge/discharge cycles, which can lead to early failure of the unhealthy cells. 
These limitations and issues are particularly problematic for traditional BESSs with fixed circuitry, which hinder their practical applications. 


% 【RBS的先进性、现状】
% Reconfigurable battery system(RBS) 解决固定电路电池组的 cask effect:系统的容量和寿命取决于状态最差的某些电池。
% 此外,不一致性也在系统运行中加剧恶化。
% 通过动态改变电路,调控或隔离不良电池,有助于提升系统整体可靠性。
% 但是,重构也增加了设计、分析和控制的难度。
% 当前,系统有成百上千的电池,平均每个电池由3~5个开关控制,形成了庞大的状态空间。
Reconfigurable Battery System (RBS), which can dynamically  switch between different circuit configurations as required, is expected to solve the above problem. 
Unlike fixed BESSs, reconfigurable circuits use additional switches to change the series/parallel relationship between batteries.
Figure 1 illustrates one of the classic architecture proposed by Visairo et al.\cite{visairoReconfigurableBatteryPack2008}.
When the longitudinal switches are cloesd and the diagonal switches are open, the batteries are connected in parallel; conversely, they are connected in series.
And any battery in the Figure can be disconnected to isolated the unhealthy one or to provide redundancy.
Based on the reconfiguration, RBSs are available to isolated the unhealthy batteries timely and balance the degradation differents between individual cells as required.
However, the reconfiguration also increases the complexity of system design and control.
Each battery in RBSs is controlled by an average of 3 to 5 switches according to reported architectures(ref:13-20).
When the system has hundreds or thousands of cells, there is a huge state space waits to solve.

\begin{figure}
    \centering
    \includegraphics[width=\textwidth]{../attachments/fig1.png}
    \caption{The physical model (a), circuit model (b) and figure model(c) of the classic architecture reported by ref. (d) The overall framework of our method.}
    \label{fig:1}
\end{figure}


% 【快速评估系统最大电流的作用和意义、文献现状】
% (一些典型结构、控制策略、评估)
% 快速评估系统最大允许电流在设计、分析和控制中起到重要作用。
% 对设计,系统的最大输出电流
% 对控制,应对故障,静态结构破坏
% 但是没有严谨研究最大电流。(直接说没有)
% del: 一些研究使用了过度的简化,不准;(具体文献?):led
% del: 只针对小数量的电池和特定的结构,不普适(具体文献,容易找到):led
The importance of the Maximum Allowable Current (MAC), defined as the maximum current that the system can output to external electrical equipment when the currents of all the batteries in the system do not exceed the specified value, is beginning to be recognised in RBS research. 
From the definition it can be seen that MAC determines the maximum output current of the RBS during normal operation at the design stage and of the reconfigured system when the failed batteries are isolated during the run-in phase. 
MAC is therefore one of the key indicators used in the study for the design and evaluation of RBSs.
% (#TODO: 一些考虑最大许用电流的构建结构的策略)
(TODO: Some literature on strategies for RBS that take MAC into account)
However, none of the existing literature on RBS provides a method for calculating the MAC.


% 【本文的主要内容和结构】
% 我们提出了快速估计RBS的算法,基于贪婪策略。填补了这一空白。
The purpose of this paper is to propose an effective and efficient algorithm based on the greedy search strategy to solve the MAC of RBSs. 
A mathematical model of MAC is constructed and the optimal solution is searched in the state space of switches using maximising the number of cells directly connected in parallel as the strategy.


% 文章组织:section 2,算法的框架和细节;section 3,案例,讨论和验证;section 4 总结。
This paper is organized as follows. 
Section II presents the mathematical model and the algorithm in detail.
Section III solves an example proposed in \cite{kimDESADependableEfficient2012}and discusses the results. 
Finally, Section IV provides concluding remarks and directions forjk future work.

\section{Method}

\subsection{graph model and circuit model}

A typical RBS circuit consists of battery cells and switches.
We first transform them into ideal components under reasonable assumptions to establish a processable circuit.
Then the circuit structure is described in matrix form and the related equations are given.


% 【电路,有向图,规定,假设】
% node,连接battery和switch,编号顺序
% edge,battery 或 switch,编号顺序
% 电池等效为恒压ub串内阻rb
% Ro,外电阻
% 关联矩阵
% x,开关状态,01变量
% 【模型,求解方法】
% 假设均一化,矩阵分块
% 在导纳矩阵非奇异的条件下
% 推导出输出电流Io和电池电流Ib
The normal equivalent circuit of a battery consists of a voltage source in series with a resistor, and a capacitor in parallel with another resistor to simulate the polarisation in batteries.
As the main aim of this research is to calculate the stable MAC offered by a given RBS architecture, only the steady-state behaviour of the circuit is taken into account, while transient effects are ignored in this research.
Thus, we equate the battery $i$ as a constant voltage source $u_{b,i}$ in series with a resistor $r_{b,i}$.
A binary variable $x_j$ is used to represent the state of the switch $j$: 0 and 1 mean open and closed respectively.
To simplify the calculation, the closed switch is considered as a resistor with a very small resistance value $r_s$, which will be treated as zero in the final result.
In the following derivation, the product of the conductance $1/r_s$ and the variable $x_j$ is used to characterize the state of the switch $j$.


Before obtaining the incidence matrix representing the structure of the circuit, a directed graph model $G(V,E)$ for the RBS is constructed in such a way that
\begin{enumerate}[(1)]
    \item the vertex set $V={v_1,v_2,\cdots,v_N}$ represents the nodes connecting batteries and/or switches, where $v_1$ and $v_N$ represent the anode and cathode of the RBS respectively;
    \item the directed edge set $E$ represents the output circuit, $N_b$ batteries and $N_s$ switches, corresponding set $E_o$, $E_b$ and $E_s$. The external electrical equipment in the output circuit is treated as one directed edge $v_n \to v_1$. The direction of the edge representing a battery is set to be from the anode to the cathode. For the edges representing switches, their directions are marked from the low node to the the high node. A negative value for the voltage drop or current solved on the edge means that the actual direction is opposite to that initially specified.
\end{enumerate}


Based on the above directed graph which has $N$ nodes and $1+N_b+N_s$ directed edges (1 refers to the output circuit), its incidence matrix $\bm{A}_{N\times (1+N_b+N_s)}$ is defined as
\begin{align}\label{eq:A}
    a_{ij}=
    \begin{cases}
        1,  & \text{edge  $j$ leaves vertex $i$},\\
        -1, & \text{edge $j$ enters vertex $i$},\\
        0,  & \text{otherwise}.
    \end{cases}
\end{align}
Since each column of $\bm{A}$ sums to zero, we delete the last line and use the reduced incidence matrix $\bm{A}_{(N-1)\times(1+N_b+N_s)}$ in the following calculation.
y splitting $E$ into $\{E_o, E_b, E_s\}$, $A$ is rewritten as follows
\begin{equation}
    \bm{A} = 
    \begin{bmatrix}
        \bm{A}_o & \bm{A}_b & \bm{A}_s
    \end{bmatrix}.
\end{equation}


$1+N_b+N_s$ edges' currents $\bm{I}_{(1+N_b+N_s)\times 1}$ and voltages $\bm{U}_{(1+N_b+N_s)\times 1}$, and $N-1$ nodes' voltages $\bm{U}_{n, (N-1)\times 1}$ have following relationships from Kirchhoffs law
\begin{align}\label{eq:Kirchhoffs_law}
    \begin{cases}
    \bm{A} \bm{I} = \bm{0}, \\
    \bm{U}        = \bm{A}^\T \bm{U}_n.
    \end{cases}
\end{align}
These directed edges are treated as generalized branches and expressed in matrix form as follows
\begin{equation}\label{eq:generalized_branches}
    \bm{I} = \bm{Y}\bm{X} \bm{U} - \bm{Y}\bm{X} \bm{U}_s +\bm{I}_s,
\end{equation}
where $\bm{I}$ and $\bm{U}$ are the column vectors about $1+N_b+N_s$ edges' current and voltage, respectively; 
$\bm{U}_s$ and $\bm{I}_s$ denote the source voltage and source current of the generalized branches, respectively;
$\bm{Y}$ is the admittance matrix of the circuit, and $\bm{X}$ is the state matrix defined as
\begin{equation}\label{eq:X}
    \bm{X} = diag(
        1,  
        \underbrace{1, \cdots, 1}_{N_b~\text{of}~1}, 
        \underbrace{1, 0 \cdots, 1}_{N_s~\text{of}~0/1}
    )
    =\begin{bmatrix}
        \bm{I} &\\
         & \bm{X}_s
    \end{bmatrix}.
\end{equation}


In addition to the equivalent circuit assumptions, we also assume that all batteries have the same internal resistance value $r_b$ and supply the same electric potential $u_s$ to simplify the model.
Then the output current $I_o$ and each battery's current $\bm{I}_b$ can be given by solving the simultaneous Equations \ref{eq:Kirchhoffs_law} and \ref{eq:generalized_branches} eventually.
Let
\begin{equation}\label{eq:Yn}
    \bm{Y}_n (\bm{X}) = \frac{1}{R_o} \bm{A}_o\bm{A}_o^\T + \frac{1}{r_b} \bm{A}_b\bm{A}_b^\T + \frac{1}{r_s}\bm{A}_s\bm{X}_s\bm{A}_s^\T,
\end{equation}
% \begin{equation}\label{eq:Yn}
%     \bm{Y}_n (\bm{X}_s) = \frac{1}{R_o} \bm{A}_o\bm{A}_o^\T + \bm{A}_b \bm{Y}_b \bm{A}_b^\T + \bm{A}_s \bm{Y}_s \bm{X}_s \bm{A}_s^\T
% \end{equation}
where $R_o$ is the equivalent resistance of the external circuit.
Then, if $\bm{Y}_n$ is an invertible matrix, 
\begin{align}
    I_o(\bm{X})      & = \frac{u_b}{R_o r_b} \bm{A}_o^\T \bm{Y}_n^{-1}(\bm{X}) \bm{A}_b \bm{I}_{N_b\times 1};\label{eq:I_o}\\
    \bm{I}_b(\bm{X}) & = \frac{u_b}{r_b^2}[\bm{A}_b^\T \bm{Y}_n^{-1}(\bm{X}) \bm{A}_b\bm{I}_{N_b \times 1}  -r_b \bm{I}_{N_b \times 1}],\label{eq:I_b}
\end{align}
% \begin{align}
%     I_o(\bm{X}_s)      & = \frac{1}{R_o}\bm{A}_o^\T\bm{Y}_n^{-1}(\bm{X}_s)\bm{A}_b\bm{Y}_b\bm{U}_b;\label{eq:I_o}\\
%     \bm{I}_b(\bm{X}_s) & = \bm{Y}_b\bm{A}_b^\T\bm{Y}_n^{-1}(\bm{X}_s)\bm{A}_b\bm{Y}_b\bm{U}_b-\bm{Y}_b\bm{U}_b,\label{eq:I_b}
% \end{align}
where $\bm{I}_{N_b\times 1}$ is a column vector with all terms being one.


% 用外电流Ib比所有电池中最大电流max(Ib)之比,表征电路的最大许用输出。是电路结构本身的性质,与电池无关。电路的线性保证了。
% 目标问题转化为【数学形式】
% Max rate
% s.t.
We use the ratio of $I_o$ and $\max (\bm{I}_b)$ to characterize the allowable current for a given RBS architecture, denoted as $\eta$.
The $\eta$ reflects the ability of the RBS architecture itself to deliver current, regardliess of the battery cells used by the RBS.
Due to the linearity of the above circuit model, the output current will vary by the same multiple if the allowable current of all batteries varies by a certain multiple.
Our problem in RBS can be formulated as
\begin{align}
        & \max \eta \label{eq:max_eta}\\
    \mathrm{s.t.}\,\, & \eta = \frac{I_o}{\max (\bm{I}_b)}, \\
        & \max (\bm{I}_b) \leq I_m,
\end{align}
where $I_m$ is the maximum allowable current of the battery; $I_o$ and $\bm{I}_b$ can be calculated by Equations \ref{eq:I_o} and \ref{eq:I_b}.
The presence of the inverse matrix $\bm{Y}_n^{-1}$ prevents us from solving \ref{eq:max_eta} directly, especially when a large number of battery cells and switches are present in the system.
We therefore propose a greedy algorithm to solve this model.


\subsection{greedy solution}
% 使用贪心算法策略求解
% 【贪心策略】
% 电池i的最短路径,短指的是路径上电池数量最少
% 贪心策略,当系统中越多的电池被以最短路径联入电路,外电路电流越大。
% 短路,检查
% 组合,逐一
First, we define the battery $i$'s shortest path ($SP_i$).
When the battery $i$ is connected to the anode $v_1$ and the cathode $v_N$ of the RBS by the path $p$ in the graph model, the distance $\omega$ of $p$ is defined by the following equation:
\begin{equation}\label{eq:weight}
    \omega(p) = N_s \cdot n_b (p) + n_s (p),
\end{equation}
where $N_s$ is the total number of switches in the system; $n_b(p)$ and $n_s(p)$ are number of batteries and switches in the path $p$ respectively.
$SP_i$ is defined as the path with the minimum $\omega$ for battery $i$.
According to the definition, $SP_i$ gives the simplest strategy by which the control of battery $i$ is achieved with a minimum of switches while minimising the influence of other batteries.
From the perspective of series/parallel, the more batteries are connected into circuit via their $SP$s, the more batteries are connected in parallel.
Since batteries can provide more total output current when connected in parallel than in series, we greedily allow as many cells as possible to be connected into to the overall circuit via their $SP$s to obtain the MAC.
The dichotomy method is also performed to faster find the right number of $SP$s.


% 【总体流程(二分查找)】
% 对于给定结构和最大许用电流通过如下步骤得到:
% (伪代码开始)
% 通过图查找,深度优先,对每个电池找最短路径
% 以二分策略选择考察电池数量N_{select}
% 通过组合,形成C^{N_select}_{N_total}种方式,对于每种方式
% 	仅将选中的电池最短路径上的开关状态设置为1
% 	求解电流,公式
%	检查各电池电流,是否短路或超过电池的最大许用电流(否,break)
% 	给出外电路电流和所有电池电流的最大值,计算比率
% (伪代码结束)
% 电池电流Ib不超过ub/rb,未短路,合法(电池电流Ib受外电路Ro影响,考虑设一个固定值Ibmax,表示电池最大许用电流)
The pseudo-code of the algorithm is as follows:
\begin{algorithm}
    \caption{Get the max available currents of a certain RBS}\label{alg:eta_RBS}
    \KwData{Directed graph model $G(V,E)$ of the RBS}
    \KwResult{$\max \eta$}
    \For{$i \in E_b$}{
        $P_i \leftarrow \{path| \text{starts at $v_1$ and ends at $v_n$} \}$\;
        $SP_i \leftarrow p_i \text{ which has the minimum}~\omega(p_i)~\text{among all}~p_i \in P_i. $
        }
    get $\bm{A}$ by Equation \ref{eq:A}\;
    \While{not yet determine $\max \eta$ }
        {
            $N_{sel} \leftarrow \text{number of selected $SP$s calculated by dichotomy}$\;
            $C_b    \leftarrow \text{set of all combinations of $N_{sel} $~batteries from $N_b$}$\;
            \For{$c_b \in C_b$}{
                % $P_{selected} \leftarrow \bigcup_{i\in c_b}SP_i$\;
                $\bm{x}_s \leftarrow \text{list of all switches' state: $x_s[j]=1$ if $ j \in \bigcup_{i\in c_b}SP_i $ else 0}$\;
                $\bm{X} \leftarrow diag[1,1,\cdots,1,\bm{x}_s] $\;
                get $\bm{Y}_n$ by Equation \ref{eq:Yn}\;
                \eIf{$\bm{Y}_n$ is invertible}{
                }{construct an effective solution}
                get $I_o$ by Equation \ref{eq:I_o}\;
                get $\bm{I}_b$ by Equation \ref{eq:I_b}\;
                \eIf{$\max(\bm{I}_b)\leq I_m$}{
                    $\eta \leftarrow I_o/\max(\bm{I}_b)$\;
                }{break}
            }
        }
\end{algorithm}

\section{Examples and discussions}
% 经典结构  visairoReconfigurableBatteryPack2008 验证我们方法的有效性,这个结构的设计目的和应用,我们的验证基于不同规模
% 建模:物理结构,图模型,电路模型
% 求解及讨论:模型的正确性,不同规模的效率


% 1. 以f4结构为例,example
% 描述试验过程,包括物理模型(结构)、图模型(有向图、SP)、电路模型(关联矩阵和状态向量)
% 描述试验结果,根据求解算法,当SPs取xxx时,电路输出电流达到MAC(图,红色表示开关闭合),各电池电流、输出电流的符号运算结果
% 讨论结果,关联矩阵A的可逆性讨论;影响输出电流的因素,对MAC定义的讨论,对“许用”的说明;
% 2. 为进一步说明模型和算法的有效性,我们对不同结构、不同电池规模进行求解
% a). 在e4、f4、e2f2结构上验证算法有效性
% b). 在f2、f4、f6、f12 上验证有效性
% (图)物理模型、求解后模型、求解结果:红色表示开关闭合
% (图)说明电池数量增加的示意图
% (表)结构、求解结果(Io,Ibmax,MAC)
% (表)电池数量、求解结果(Io,Ibmax,MAC)
% 准确性:算法MAC求解结果与电路定律符合
% 正确性:模型将电池的物理模型等效为恒压源和电阻,外电路等效为电阻,验证该假设的合理性——分析ub、rb等对结果的影响
% 3. 该方法的优点和未来改进方向:
% a) 可以获得电路的所有信息,用于其他可重构电路指标的计算,如输出电压等
% b) 可对电池等效模型进行修改,如引入电容,分析暂态


In this section, we evaluate the algorithm and models for MAC mentioned earlier, based on some previously reported structures. 
Firstly, we provide a specific example for the algorithm and models proposed in Section II using the structure proposed by Visairo and discuss the model and computation process.
Next, we further verify the applicability of our method on RBSs with different structures and sizes. 
Finally, we highlight the advantages of our method and suggest directions for future improvements.


\subsection{An specific example}

\begin{figure}[htbp]
  \centering
  \begin{subfigure}[b]{0.45\textwidth}
    \includegraphics[width=\textwidth]{../attachments/f4-phy.png}
    \caption{}
    \label{fig:f4-phy}
  \end{subfigure}
  \hspace{0.05\textwidth}
  \begin{subfigure}[b]{0.45\textwidth}
    \includegraphics[width=\textwidth]{../attachments/f4-gra.png}
    \caption{}
    \label{fig:f4-gra}
  \end{subfigure}
  \\
  \begin{subfigure}[b]{0.45\textwidth}
    \includegraphics[width=\textwidth]{../attachments/f-dege-4-modify.png}
    \caption{}
    \label{fig:f4-circ}
  \end{subfigure}
  \hspace{0.05\textwidth}
  \begin{subfigure}[b]{0.45\textwidth}
    \includegraphics[width=\textwidth]{../attachments/f-dege-mac-4.png}
    \caption{}
    \label{fig:f4-mac}
  \end{subfigure}
    
  \caption{the physical model (a), circuit model (b) and figure model(c) of the reconfigurable architecture propoesd by visairo \cite{visairoReconfigurableBatteryPack2008}(d) is the result}
  \label{fig:f4-all}
\end{figure}


The reconfigurable architecture proposed by Visairo et al.\cite{visairoReconfigurableBatteryPack2008} is shown in Figure \ref{fig:f4-phy}.
In this architecture, each cell is controlled by about 3 switches on average.
Excepted for two switches ($S_1$ and $S_{13}$ in Figure \ref{fig:f4-phy}) controlling the total circuit, the vertical switches (e.g. $S_2$ and $S_9$ in Figure \ref{fig:f4-phy}) allow the batteries to be directly connected to the main circuit, and the switches($S_6$ for example) in the diagonal direction can realize connecting the batteries in series.
Thus, it can dynamically change the output voltage and current as needed.
The architecture in Figure \ref{fig:f4-phy} only has 4 batteries, but in practice it can add new branches containing batteries and switches to deal with large-scale batteries\cite{kimDependableEfficientScalable2010}.


The graph in Figure \ref{fig:f4-gra} represents a graphical model based on the physical model(Figure \ref{fig:f4-phy}). 
The model is presented in the form of a directed graph, capturing the connection relationship between the battery and switch in the circuit. 
Vertex 1 and 12 represent the ancode and cathode of the RBS, respectively; 
the remaining vertices represent the connection nodes between the batteries and switches. 
Battery 1, 2, 3, and 4 are represented by green directed edges in the graph, while switches are represented by gray directed edges with opposite directions. 
The external electrical appliance is considered as a directed edge from node 12 to node 1. 
In the graph model, the directionality of the edges is strictly specified to ensure that there is no reverse flow of current in the battery and external electrical appliance, which could damage the appliance. 


According to Equation \ref{eq:weight}, the weight of the edge corresponding to the battery is the total number of switches, which is 13; 
the weight of the edge corresponding to the switch is 1. 
By setting the weight in this way, on the one hand, it avoids the simultaneous occurrence of multiple batteries in a single SP, weakening the influence between different batteries; 
on the other hand, it minimizes the number of switches on the path, which means that only a few switches (i.e., the switches on the SP) need to be closed to connect the battery to the main circuit. 
By using the depth-first search algorithm, the SP of battery 1 can be calculated as the node list $[1, 2, 3, 7, 11, 12]$, and the SPs of the other batteries are shown in Figure \ref{fig:f4-gra}. 
The SPs provide a basis for selecting variables when solving the MAC through the circuit model in the future.


The equivalent circuit model, as shows in Figure \ref{fig:f4-circ}, can be obtained based on the assumptions in Section II. 
In this model, the batteries and switches are treated as branches, where the battery is regarded as a branch consisting of a voltage source $u_b$ in series with a resistor $r_b$, and the switch is regarded as a branch controlled by a state variable $x_s$ to short or open the circuit.
Difference from the directed graph model, only one directed edge is used to represent the branch in the circuit, regardless of whether it represents a battery or a switch. 
The direction of the edge is defined as pointing from the node with the smaller number to the  larger . 
The direction of the edge is only a preset direction used to calculate the current and voltage. 
When the computed current or voltage value is negative, it indicates that the direction of the battery or potential difference is opposite to the preset direction. 
Based on the node-edge relationship in Figure \ref{fig:f4-circ}, the incidence matrix $\bm{A}$: can be obtained according to Equation \ref{eq:A}:
% {\setlength{\arraycolsep}{4pt}
% \begin{equation}
% \begin{array}{cc}
%     &  \begin{array}{c c ccc cccccccccc} & e_o &e_{b1}  &e_{b2} & e_{b3} & e_{s1} & e_{s2} & e_{s3} & e_{s4} & e_{s5} & e_{s6} & e_{s7} & e_{s8} & e_{s9} & e_{s10} \end{array}\\
%         \begin{array}{c} v_1\\v_2\\v_3\\v_4\\v_5\\v_6\\v_7\\v_8\\v_9\end{array} & \left[
%     \begin{array}{c|ccc|cccccccccc}
%         -1  &0 &0 &0   &1 &0 &0 &0 &0 &0 &0 &0 &0 &0\\
%         0   &0 &0 &0   &-1&1 &1 &1 &0 &0 &0 &0 &0 &0\\
%         0   &1 &0 &0   &0 &-1&0 &0 &1 &0 &0 &0 &0 &0\\
%         0   &0 &1 &0   &0 &0 &-1&0 &0 &1 &0 &0 &0 &0\\
%         0   &0 &0 &1   &0 &0 &0 &-1&0 &0 &0 &0 &0 &0\\
%         0   &-1&0 &0   &0 &0 &0 &0 &0 &0 &1 &0 &0 &0\\
%         0   &0 &-1&0   &0 &0 &0 &0 &-1&0 &0 &1 &0 &0\\
%         0   &0 &0 &-1  &0 &0 &0 &0 &0 &-1&0 &0 &1 &0\\
%         0   &0 &0 &0   &0 &0 &0 &0 &0 &0 &-1&-1&-1&1\\
%     \end{array} \right]
% \end{array},
% \end{equation}
% }
{\setlength{\arraycolsep}{2pt}
\begin{equation}
\begin{array}{cc}
    &  \begin{array}{c c cccc ccccccccccccc} & e_o &e_{b1}  &e_{b2} & e_{b3} & e_{b4} & e_{s1} & e_{s2} & e_{s3} & e_{s4} & e_{s5} & e_{s6} & e_{s7} & e_{s8} & e_{s9} & e_{s10} & e_{s11} & e_{s12} & e_{s13} \end{array}\\
        \begin{array}{c} v_1\\v_2\\v_3\\v_4\\v_5\\v_6\\v_7\\v_8\\v_9\\v_{10}\\v_{11}\\v_{12}\end{array} & \left[
    \begin{array}{c|cccc|ccccccccccccc}
    -1  &  0  &  0  &  0  &  0  &  1  &  0  &  0  &  0  &  0  &  0  &  0  &  0  &  0  &  0  &  0  &  0  &  0\\
     0  &  0  &  0  &  0  &  0  & -1  &  1  &  1  &  1  &  1  &  0  &  0  &  0  &  0  &  0  &  0  &  0  &  0\\
     0  &  1  &  0  &  0  &  0  &  0  & -1  &  0  &  0  &  0  &  1  &  0  &  0  &  0  &  0  &  0  &  0  &  0\\
     0  &  0  &  1  &  0  &  0  &  0  &  0  & -1  &  0  &  0  &  0  &  1  &  0  &  0  &  0  &  0  &  0  &  0\\
     0  &  0  &  0  &  1  &  0  &  0  &  0  &  0  & -1  &  0  &  0  &  0  &  1  &  0  &  0  &  0  &  0  &  0\\
     0  &  0  &  0  &  0  &  1  &  0  &  0  &  0  &  0  & -1  &  0  &  0  &  0  &  0  &  0  &  0  &  0  &  0\\
     0  & -1  &  0  &  0  &  0  &  0  &  0  &  0  &  0  &  0  &  0  &  0  &  0  &  1  &  0  &  0  &  0  &  0\\
     0  &  0  & -1  &  0  &  0  &  0  &  0  &  0  &  0  &  0  & -1  &  0  &  0  &  0  &  1  &  0  &  0  &  0\\
     0  &  0  &  0  & -1  &  0  &  0  &  0  &  0  &  0  &  0  &  0  & -1  &  0  &  0  &  0  &  1  &  0  &  0\\
     0  &  0  &  0  &  0  & -1  &  0  &  0  &  0  &  0  &  0  &  0  &  0  & -1  &  0  &  0  &  0  &  1  &  0\\
     0  &  0  &  0  &  0  &  0  &  0  &  0  &  0  &  0  &  0  &  0  &  0  &  0  & -1  & -1  & -1  & -1  &  1\\
    \end{array} \right]
\end{array},
\end{equation}
}
where the rows correspond to vertexes in the graph model, and the first column, second to fourth columns, and last ten columns correspond to external electrical equipment, 4 batteries, and 13 switches, respectively.
According to the above classification of the columns, the matrix $\bm{A}$ can be divided into $\bm{A}_o$, $\bm{A}_b$, and $\bm{A}_s$.


The state matrix $\bm{X}$ is determined by the switches' state, that is, the specific configuration of the RBS.
For example, when switch
% $S_1$, $S_2$, $S_3$, $S_4$ , $S_7$, $S_8$, $S_9$ and $S_{10}$  % for f3
$S_1$, $S_2$, $S_3$, $S_4$, $S_5$, $S_9$, $S_{10}$, $S_{11}$, $S_{12}$ and $S_{13}$ 
are closed, and switch $S_6$, $S_7$ and $S_8$ are open, that is , battery $B_1$, $B_2$, $B_3$ and $_4$ are connected in parallel to supply power to the external circuit, the state matrix $\bm{X}$ is given by
\begin{equation}
%     \bm{X} = diag(
%         1,  
%         \underbrace{1,1,1}_{\text{batteries}}, 
%         \underbrace{1,1,1,1,0,0,1,1,1,1}_{\text{switches}}
%     ).
    \bm{X} = diag(
        1,  
        \underbrace{1,1,1,1}_{\text{batteries}}, 
        \underbrace{1,1,1,1,1,0,0,0,1,1,1,1,1}_{\text{switches}}
    ).
\end{equation}


When using Algorithm 1 to solve the MAC, we may encounter the situation where the matrix $\bm{Y}_n$ is not full rank, and we cannot directly obtain its inverse matrix. 
This is because some switches are open, causing some branches with voltage sources to be independent of the main circuit and form new circuits. 
These circuits have infinite possible values for potential difference between them, since they are not connected to each other. 
In other words, there are theoretically infinite solutions for potential. 
However, all of these solutions will satisfy the voltage and current laws required by the equation. 
On the one hand, for the main circuit, since the model specifies node 12 as the reference point for potential at the beginning (i.e., the associated matrix $\bm{A}$ is reduced by one row corresponding to node 12), the submatrix of $\bm{Y}_n$ formed by the edges on the main circuit path must be full rank, so there is no problem as described above. 
On the other hand, since the new circuit is independent of the main circuit, it will not affect the output currents of the main circuit and the battery currents. 
Therefore, any solution can be used for subsequent calculations.
Here, we can choose the solution where the potential at the node with the smallest number in the independent branch is 0.

% MAC of RBS_f(4): 4.00
% Io_ideal: (4*ub)/(4*Ro + rb)
% Ib_ideal(1): ub/(4*Ro + rb)
% Ib_ideal(2): ub/(4*Ro + rb)
% Ib_ideal(3): ub/(4*Ro + rb)
% Ib_ideal(4): ub/(4*Ro + rb)
% those switches are close: 1 2 3 4 5 9 10 11 12 13 
The final calculation result of $\eta$ for the structure in Figure \ref{fig:f4-phy} is 4. 
The corresponding currents of output and each battery are shown in Table \ref{tab:1}. 
The red edges in Figure \ref{fig:f4-mac} represent the closed switches.
\begin{table}[h]
    \caption{TODO}
    \label{tab:1}
    \begin{tabular}{ccc}
        \hline
        output current $I_o$       & battery current $\bm{I}_b$       & $\eta$        \\ 
        \hline\\
        $\displaystyle\frac{4u_b}{4R_o + r_b}$ &  $\displaystyle\left[\frac{u_b}{4R_o + r_b},\frac{u_b}{4R_o + r_b},\frac{u_b}{4R_o + r_b},\frac{u_b}{4R_o + r_b}\right]$   & $4.00$ \\ 
        \\
        \hline
    \end{tabular}
\end{table}
For this structure, when all batteries are connected to the main circuit through the switches in the vertical direction in Figure \ref{fig:f4-phy} and the batteries are connected in parallel, the output current is maximum, which is the sum of the currents of the four batteries. 
From the specific results of the output current $I_o$ and battery currents $\bm{I}_b$, both are related to the battery voltage $u_b$, internal resistance $r_b$, and external circuit resistance $R_o$. 
This means that even for the same structure, the output current will change when the external circuit  or the battery specifications  change. 
By dividing the output current by the sum of the battery currents, we obtain $\eta$, in which the effects of these variables cancel out.
Thanks to the linearity of the studied structure, when the battery specifications change, the output current also changes proportionally, such as doubling the battery voltage would also double the output current. 
This variable can reflect the characteristics of the structure itself, regardless of the batteries used. 
Therefore, using $\eta$ as the metric for the MAC is better than using the output current directly.

\subsection{Difference structures and sizes}

To further demonstrate the effectiveness of the model and algorithm, we solve for different structures and RBSs of different sizes.

TODO: 
% 贪婪算法

\begin{figure}[htbp]
  \centering
  \begin{subfigure}[b]{0.3\textwidth}
    \includegraphics[width=\textwidth]{../attachments/f-dege-mac-4.png}
    \caption{}
    \label{fig:f4-mac}
  \end{subfigure}
  \hspace{0.05\textwidth}
  \begin{subfigure}[b]{0.3\textwidth}
    \includegraphics[width=\textwidth]{../attachments/e-dege-mac-4.png}
    \caption{}
    \label{fig:e4-mac}
  \end{subfigure}
  \hspace{0.05\textwidth}
  \begin{subfigure}[b]{0.45\textwidth}
    \includegraphics[width=\textwidth]{../attachments/e2f2-dege-mac.png}
    \caption{}
    \label{fig:e2f2-mac}
  \end{subfigure}
    
  \caption{TODO}
  \label{fig:d-structure}
\end{figure}


\begin{table}[h]
    \caption{TODO}
    \label{tab:d-structure}
    \begin{tabular}{cccc}
        \hline
        structure &  output current $I_o$       & battery current $\bm{I}_b$       & $\eta$        \\ 
        \hline\\
        f &  $\displaystyle\frac{4u_b}{4R_o + r_b}$ &  $\displaystyle\left[\frac{u_b}{4R_o + r_b},\frac{u_b}{4R_o + r_b},\frac{u_b}{4R_o + r_b},\frac{u_b}{4R_o + r_b}\right]$   & $2.00$ \\ 
        \\
        e &  $\displaystyle\frac{u_b}{R_o + r_b}$ &  $\displaystyle\left[\frac{u_b}{R_o + r_b},0,0,0\right]$   & $4.00$ \\ 
        \\ 
        e2f2 &  $\displaystyle\frac{2u_b}{2R_o + r_b}$ &  $\displaystyle\left[\frac{u_b}{2R_o + r_b},\frac{u_b}{2R_o + r_b},0,0\right]$   & $6.00$ \\ 
        \\
        \hline
    \end{tabular}
\end{table}

\begin{figure}[htbp]
  \centering
  \begin{subfigure}[b]{0.3\textwidth}
    \includegraphics[width=\textwidth]{../attachments/f-dege-mac-2.png}
    \caption{}
    \label{fig:f2-mac}
  \end{subfigure}
  \hspace{0.05\textwidth}
  \begin{subfigure}[b]{0.3\textwidth}
    \includegraphics[width=\textwidth]{../attachments/f-dege-mac-4.png}
    \caption{}
    \label{fig:f4-mac}
  \end{subfigure}
  \hspace{0.05\textwidth}
  \begin{subfigure}[b]{0.45\textwidth}
    \includegraphics[width=\textwidth]{../attachments/f-dege-mac-6.png}
    \caption{}
    \label{fig:f6-mac}
  \end{subfigure}
    
  \caption{TODO}
  \label{fig:d-sizes}
\end{figure}

\begin{table}[h]
    \caption{TODO}
    \label{tab:d-structure}
    \begin{tabular}{cccc}
        \hline
        size &  output current $I_o$       & battery current $\bm{I}_b$       & $\eta$        \\ 
        \hline\\
        2 &  $\displaystyle\frac{2u_b}{2R_o + r_b}$ &  $\displaystyle\left[\frac{u_b}{2R_o + r_b},\frac{u_b}{2R_o + r_b},\frac{u_b}{2R_o + r_b},\frac{u_b}{2R_o + r_b}\right]$   & $4.00$ \\ 
        \\
        4 &  $\displaystyle\frac{4u_b}{4R_o + r_b}$ &  $\displaystyle\left[\frac{u_b}{4R_o + r_b},\frac{u_b}{4R_o + r_b},\frac{u_b}{4R_o + r_b},\frac{u_b}{4R_o + r_b}\right]$   & $4.00$ \\ 
        \\
        6 &  $\displaystyle\frac{6u_b}{6R_o + r_b}$ &  $\displaystyle\left[\frac{u_b}{6R_o + r_b},\frac{u_b}{6R_o + r_b},\frac{u_b}{6R_o + r_b},\frac{u_b}{6R_o + r_b}\right]$   & $4.00$ \\ 
        \\
        \hline
    \end{tabular}
\end{table}

\subsection{Advantages and future improvements}



Our proposed method can obtain comprehensive information about the battery system, including the voltage, current, power, and energy of each battery, as well as the output voltage and current of the system.
This information can be used for various reconfigurable circuit metrics, such as output voltage regulation, power balancing, and fault diagnosis. 
By using this method, engineers and researchers can obtain a better understanding of the behavior and performance of battery systems, and optimize their designs and operations accordingly.

To further enhance the usefulness of our method, we can modify the equivalent battery model by introducing capacitance to analyze the transient behavior of the battery system. 
This modification can capture the dynamic response of the battery system to sudden changes in load or input voltage, and enable more accurate simulations of battery systems under transient conditions.
By incorporating these models and techniques, we can achieve more accurate and realistic simulations of battery systems, and enable more advanced analysis and design of battery systems.


\section{Conclusion}

In conclusion, we have successfully addressed the problem of estimating the maximum allowable current (MAC) for RBS. First, we established a directed graph model to describe the connection relationship between the battery and the switch based on the physical model and solved the shortest path (SP) for the battery to connect to the main circuit. Then, we established a circuit model by equivalently representing the battery as a series voltage source and resistance and used switch state variables (X_s) to describe RBS reconstruction. Based on a greedy algorithm, we used SP to provide a search strategy for X_s and proposed using the ratio of output current to maximum battery current to describe MAC. The effectiveness of this method has been verified on RBS with different structures and scales. Our model can provide information on current and voltage in each branch of the circuit and is expandable.



\bibliographystyle{ieeetr}
\bibliography{../attachments/my_ref}

\end{document}
