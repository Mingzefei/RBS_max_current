\documentclass[12pt,american]{scrartcl}
\usepackage{babel}
\usepackage[babel]{microtype}
\usepackage[babel]{csquotes}

\usepackage[journal={Space: Science \& Technology},
			manuscript={SPACE-D-23-00082},
			editor={Dr. Tian}]{reviewresponse}

\usepackage[T1]{fontenc}
\usepackage{lmodern}
%\usepackage{newcent}
%\usepackage[scaled]{beramono}

\usepackage[]{biblatex}
\bibliography{2-SST_review.bib}

\usepackage{hyperref}

\title{Maximum Allowable Current Determination of RBS By Using a Directed Graph Model and Greedy Algorithm}
\author{Dr. Cheng Qian, on behalf of all authors}


\begin{document}
\maketitle

% Cover Letter
\noindent Dear \editorname,

On behalf of my co-authors, we thank you very much for giving us an opportunity to revise our manuscript, we appreciate editor and reviewer very much for their positive and constructive comments and suggestions on our manuscript entitled \enquote{\thetitle}(ID:\manuscript).

We have studied reviewer's comments carefully and have made revision which marked in red in the paper. We have tried our best to revise our manuscript according to the comments. Specially, this manuscript has been retouched by the professional organization, to further improve the readability. Attached please find the revised version, which we would like to submit for your kind consideration.

We would like to express our great appreciation to you and reviewers for comments on our paper. Looking forward to hearing from you.

Thank you and best regards.

\vspace{1.2em}

\noindent Sincerely,

\vspace{1.5em}

\noindent \theauthor


% Reviewer 1
\reviewer
% Comment 1
\begin{revcomment}
	The authors should explain the most important achievements of the proposed method quantitatively, in the abstract.
\end{revcomment}
\begin{revresponse}
	We appreciate the reviewer's valuable feedback and have carefully considered their suggestion. In response, we have re-written the relevant content in the abstract to provide a quantitative analysis of the most important achievements of our proposed method.

	The revised abstract now highlights that our method not only achieves the same results as the traversal method but also demonstrates significantly improved computational efficiency. Specifically, our method achieves computational efficiency improvements ranging from 3000 to 75000 times, primarily depending on the number of switches involved. This enhancement in computational efficiency is a key advantage of our method.

	Furthermore, our method stands out for its ability to accurately calculate the MAC of RBSs with arbitrary configurations, even in scenarios with random isolated batteries. This capability expands the applicability of our method to a wider range of scenarios and enhances its practical utility.

	We believe that these quantitative achievements, as described in the revised abstract, effectively demonstrate the effectiveness and superiority of our proposed method.
\end{revresponse}

% Comment 2
\begin{revcomment}
	The literature review in the introduction section is very short, and the related works, especially the works published in recent years, have not been well reviewed and compared, and the conclusions about the existing research gaps have not been presented.
\end{revcomment}
\begin{revresponse}
	
\end{revresponse}

% Comment 3
\begin{revcomment}
	It is necessary for the authors to clearly state research contribution and achievements as bullet points at the end of the Introduction section.
\end{revcomment}
\begin{revresponse}
	
\end{revresponse}

% Comment 4
\begin{revcomment}
	The authors need to present the complexity of their proposed method and compare it with some other state-of-the-art or successful classic methods.
\end{revcomment}
\begin{revresponse}
	
\end{revresponse}

% Comment 5
\begin{revcomment}
	The authors don't discuss the limitations of the study correctly.
\end{revcomment}
\begin{revresponse}
	
\end{revresponse}


% Comment 6
\begin{revcomment}
	Some typos should be double check.
\end{revcomment}
\begin{revresponse}
	
\end{revresponse}

% Comment 7
\begin{revcomment}
	The author should explain more why solution quality of their proposed approach is much better than the others?
\end{revcomment}
\begin{revresponse}
	
\end{revresponse}

% Comment 8
\begin{revcomment}
	Authors should mention some novel works in the field in the introduction, specially refer to this 2023 reference: An efficient lightweight algorithm for scheduling tasks onto dynamically reconfigurable hardware using graph-oriented simulated annealing, which uses graph-based method. Mention and refer to it in the introduction section.
\end{revcomment}
\begin{revresponse}
	
\end{revresponse}

% Comment 9
\begin{revcomment}
	Authors need to explain about the accuracy, sufficiency and reliability of their results? How do they verify and validate the results?
\end{revcomment}
\begin{revresponse}
	
\end{revresponse}



% \begin{revcomment}
% 	Everything else is really good.
% \end{revcomment}
% \begin{revresponse}
% 	We totally agree. We also added the following to the new version of the manuscript
% 	\begin{changes}
% 		This really important sentence was added to the paper.
% 	\end{changes}
% \end{revresponse}

\end{document}