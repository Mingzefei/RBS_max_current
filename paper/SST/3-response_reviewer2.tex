% Reviewer 2
\reviewer
% Comment 1
\begin{revcomment}
  The latest related works need to be reviewed carlfully please, expecicially the works published in recent year 2022,2023.
\end{revcomment}
\begin{revresponse}

We concur with the reviewer's assertion that it is important for our paper to undergo a thorough review of the most recent relevant literature. 
We have carefully scrutinized the pertinent works from the past five years and made appropriate revisions to the manuscript.
In the "Introduction" section, we provide an overview of the challenges posed by the complex RBS structures in hardware design. 
Furthermore, the estimation and control of the system state of RBSs have been receiving increasing attention in recent time.
Consequently, we have chosen several corresponding methodologies aimed at optimizing the system's performance.


Here is the specific modification in the introduction:
\begin{changes}
  \ExecuteMetaData[3-main]{reviewer2-comment1}
\end{changes}

\end{revresponse}

% Comment 2
\begin{revcomment}
  The greedy algorithm and the brute force algorithm are compared in this paper, but the advantages and disadvantages of this algorithm compared with other algorithms cannot be determined.
\end{revcomment}
\begin{revresponse}

Thanks to the suggestion of the reviewer! 
We have revised the "Discussion" subsection and included a specific part to analyze the advantages and disadvantages of the proposed greedy algorithm in comparison to other algorithms. 
Based on our discussion and comparison, the proposed greedy algorithm demonstrates a significant advantage in terms of its effectiveness and efficiency.
It is also capable of handling RBSs with diverse structures. 
However, this algorithm may encounter challenges when dealing with large-scale problems due to its exponential time complexity.
Furthermore, the simplification of the derivation by assuming that all batteries are identical may introduce a slight bias to the MAC due to variations in open-circuit voltage $u_b$ and internal resistance $r_b$ in reality.
We have also provided a solution to address this issue.


The relevant content has been added to the "Pros and cons analysis" part within the "Discussion" subsection. 
Once again, we appreciate this constructive comment from the reviewer.

\end{revresponse}

% Comment 3
\begin{revcomment}
  This paper mainly applies to the four-battery system, but the usability of other structures of RBS should also be discussed.
\end{revcomment}
\begin{revresponse}

Thanks for the valuable feedback. 
We have thoroughly considered the comment made by the reviewer.
In response, we have supplemented the case study with a series of experiments on RBSs with variant batteries. 
Overall, the proposed greedy algorithm has been applied to RBSs with three different structures, variant batteries, and scenarios involving random isolated batteries. 
The correctness and efficiency of the proposed greedy algorithm have been verified through the comparison with other algorithms.

The relevant content has been added to the "Case study" section. 
We hope it can address the reviewer's concerns.

\end{revresponse}